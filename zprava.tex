\documentclass[slovak]{article}
\usepackage[dvips]{graphicx}        % to include images
\usepackage{pslatex}	    % to use PostScript fonts
\usepackage[T1]{fontenc}
\usepackage[utf8]{inputenc}
\usepackage{pslatex}

\begin{document}

\begin{center}
\bf Semestrálna práca MI-PAR 2013/2014:\\[5mm]
    Paralelný algoritmus pre rozklad obdĺžnikov\\[5mm]
       Jakub Melezínek\\
       Martin Klepáč\\[2mm]
\today
\end{center}

\section{Definícia problému}

Našou úlohou bolo vytvoriť program, ktorý implementuje usporiadanie obdĺžnikov so zachovaním minimálneho celkového obvodu týchto obdĺžnikov.

Obsah obdĺžnika je daný hodnotou uloženou v 2D mriežke, pričom tento element musí byť súčasťou obdĺžnika s daným obsahom.

Jednotlivé obdĺžniky sú vzájomne disjunktné až na spoločné vrcholy a hrany, pričom zjednotenie všetkých obdĺžnikov pokrýva pôvodnú mriežku.

Riešení, ako rozdeliť mriežku na jednotlivé obdĺžniky, môže byť viac -- v takom prípade hľadáme riešenie s minimálnym obvodom, ale zároveň riešenie nemusí existovať.


\section{Formát vstupu, výstupu}

Formálne, vstup vyjadríme pomocou \\ 

\emph{a, b} = prirodzené čísla predstavujúce rozmery mriežky

\emph{H[1..a][1..b]} = mriežka

\emph{n} = prirodzené číslo predstavujúce počet obdĺžnikov vo vnútri mriežky \\

Výstupom algoritmu je okrem celkového obvodu dielčích obdĺžnikov vyfarbená mriežka, t.j. mriežka, v ktorej každému elementu je priradené písmeno abecedy, ktoré reprezentuje číslo obdĺžnika, ktorého je bod súčasťou.

\end{document}
