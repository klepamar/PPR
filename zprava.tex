\documentclass[slovak]{article}
\usepackage[dvips]{graphicx} % to include images
\usepackage{pslatex}	     % to use PostScript fonts
\usepackage[T1]{fontenc}
\usepackage[utf8]{inputenc}
\usepackage{pslatex}

\usepackage{tabularx} % tabulky na celu sirku strany
\usepackage{graphicx} % graphics files inclusion
\usepackage{multirow}
\usepackage{pdfpages}

% README
% ======
% kompilacia: $ pdflatex zprava.tex (pre istotu dvakrat za sebou kvoli pouzitiu spatnych referencii)
% obrazky: vo formate .pdf - napriklad konverziou z .eps (s tym dokaze pracovat gimp) prikazom 'epstopdf'; gimp zaroven dokaze priamo exportovat do pdf-ka

% v pripade, ze latex defaultne spatne deli slova
\hyphenation{Za-u-jí-ma-vej-šou}
\hyphenation{u-miest-niť}
\hyphenation{žia-da-li}

\begin{document}

\title{Semestrálna práca MI-PAR 2013/2014: \\[5mm] Paralelný algoritmus pre rozklad obdĺžnikov}
\author{Jakub Melezínek \\[2mm]Martin Klepáč}
\date{\today}

\maketitle

\section{Definícia problému}

Našou úlohou bolo vytvoriť program, ktorý implementuje usporiadanie obdĺžnikov v 2D mriežke so zachovaním minimálneho celkového obvodu týchto obdĺžnikov.

Obsah obdĺžnika je daný hodnotou uloženou v 2D mriežke, pričom tento element musí byť súčasťou obdĺžnika s daným obsahom.

Jednotlivé obdĺžniky sú vzájomne disjunktné až na spoločné vrcholy a hrany, pričom zjednotenie všetkých obdĺžnikov pokrýva pôvodnú mriežku.

Riešení, ako rozdeliť mriežku na jednotlivé obdĺžniky, môže byť viac -- v takom prípade hľadáme riešenie s minimálnym obvodom -- zároveň ale riešenie nemusí existovať.


\section{Formát vstupu, výstupu}

Formálne, vstup vyjadríme pomocou

\begin{itemize}

\item \emph{a, b} = prirodzené čísla predstavujúce rozmery mriežky

\item \emph{H[1..a][1..b]} = mriežka

\item \emph{n} = prirodzené číslo predstavujúce počet obdĺžnikov vo vnútri mriežky

\end{itemize}

Výstupom algoritmu je okrem celkového obvodu dielčích obdĺžnikov vyfarbená mriežka, t.j. mriežka, v ktorej každému elementu je priradené písmeno abecedy, ktoré jednoznačne identifikuje obdĺžnik, ktorého je bod súčasťou.


\section{Implementácia sekvenčného riešenia}

Primárny cieľ sekvenčného riešenia, na ktorom ďalej staviame v paralelnej implementácii, spočíva v nájdení a následnom prehľadaní celého stavového priestoru množiny potenciálnych riešení. 

Stavový priestor v našom prípade rozumieme množinu mriežok, v ktorých postupne spracúvame obdĺžniky s dvojicou parametrov

\begin{enumerate}

\item \emph{shape} - veľkosť obdĺžnika (dĺžka x šírka)

\item \emph{position} - pozícia v mriežke

\end{enumerate}

Očividne, \emph{shape} obdĺžnika je určený jeho plochou, zatiaľ čo \emph{position} je vlastnosťou mriežky a ostatných obdĺžnikov v nej existujúcich - napr. obdĺžnik nedokážem vložiť do mriežky, pokiaľ jeden z jeho rozmerov presahuje veľkosť mriežky alebo je okolie obdĺžnika posiate inými, už zafixovanými obdĺžnikmi.

Popis sekvečného riešenia v skratke: procesor zo zásobníka mriežok vezme obdĺžnik, zistí, či má \emph{shape} (v prípade negatívnej odpovede vygeneruje všetky dvojice \emph{a}, \emph{b} tak, aby ich súčin bol rovný očakávanej ploche). V ďalšom kroku procesor zistí, či daný obdĺžnik má stanovenú \emph{position} v mriežke - ak nie, na zásobník uloží všetky prípustné možnosti s ohľadom na veľkosť mriežky a susedné obdĺžniky. Procesor pokračuje do spracovania posledného obdĺžnika alebo do nájdenia prvého obdĺžnika, ktorý nemožno umiestniť do mriežky. V prípade, ak sa všetky obdĺžniky podarilo umiestniť do mriežky, procesor porovná výsledok s doterajším minimom a zo zásobníka vezme ďalšiu mriežku a opakuje takto popísanú akciu.

V nami implementovanom triviálnom sekvenčnom riešení neuplatňujeme orezávanie neperspektívnych ciest - každá mriežka dobehne do konca v prípade existencie rozdelenia obdĺžnikov a až následne sa porovná výsledný obvod s doterajším minimom.

Trvanie sekvenčného výpočtu na výpočtovom klastri star.fit.cvut.cz je znázornené v tabuľke \ref{tab:sek}. Výsledný sekvenčný čas je určený ako aritmetický priemer trojice meraní. Rovnakú metodiku sme použili aj pri popisovaní výsledkov paralelného behu.

% tabulka na celu sirku strany
	\begin{table}\centering
		\begin{tabularx}{\textwidth}{|X|X|X|}
			\hline                        
			\textbf{Veľkosť mriežky} & \textbf{T(n) [s]} \\ \hline
			\textbf{15x15} 	& 29 \\ \hline
			\textbf{20x20} 	& 473 \\ \hline
			\textbf{21x21} 	& 571 \\ \hline
			\textbf{23x23} 	& 983 \\ \hline
		\end{tabularx}
	\caption{Trvanie sekvenčného výpočtu}
	\label{tab:sek}
	\end{table}

\section{Príklad zadania a výsledku}

Program spúšťame s parametrom -f, ktorý udáva cestu k súboru obsahujúce vstupné dáta. Pre triviálny vstup o veľkosti mriežky 5x5 dostávame výstup zobrazený na obrázku \ref{fig:sek}.

% !h = here, nepouzi floating prostredie, ktore latex defaultne vyuziva
% latex nedokazal spravne urcit velkost vstupneho obrazku, preto som rozmery musel dodat a nasledne zmensit obrazok na 40% povodneho
	\begin{figure}[!h]\centering
	\includegraphics[scale=0.40,natwidth=503,natheight=648]{./sequential.pdf}
	\caption{Príklad výstupu pre sekvenčnú úlohu}\label{fig:sek}
	\end{figure}

\section{Implementácia paralelného riešenia}

Paralelní algoritmus začíná tím, že Master načte vstupní data a začne
řešit úlohu. Ve chvíli, kdy má dostatek práce na odeslání práce alespoň
jednomu procesoru, rozdělí zásobník podélně, přičemž se z každé úrovně
stromu vezme jen jeden prvek. Oddělenou část zásobníku Master neblokujícím
způsobem odesílá nenastartovanému procesoru. Jednotlivé procesory jsou
tedy startovány postupně a obdrží dostatek práce. Procesory bez prvotní
práce (neaktivní/čekající) čekají na zprávu s prací od Mastera a mezitím
obsluhují i ostatní zprávy.

Nastartovaný procesor řeší úloho pomocí algoritmu ze sekvenčního řešení.
Navíc je implementováno ořezávání stavového prostoru podle nejlepšího
známého lokálního řešení. Procesy toto lokální nejlepší řešení posílají
zároveň s odesílanou prací a je tedy zajištěna určitá propagace globálního
nejlepšího řešení.

Pokud procesoru dojde práce, odešle token pokud ho drží (procesor si
přijatý token drží do chvíle než má prázdný zásobník, aby nedocházelo ke
zbytečnému posílání zpráv). Poté zažádá náhodného dárce o práci a dokud
neobdrží zprávu s prací, obsluhuje přijaté zprávy a případně žádá nového
dárce. Dárce se pokusí rozdělit zásobník u dna. Pokud se mu to povede,
odesílá práci (a své lokální nejlepší řešení). Pokud nemá dostatek práce,
aby mohl být zásobník rozdělen, odesílá zprávu bez práce a žádající hledá
nového dárce.

Aktivní procesor každých 50 vyřešených DFS (vnější cyklus algoritmu -
umístění jednoho obdélníku) obsluhuje přijaté žádosti o práci. Ostatní
zprávy obsluhuje jen v případě prázdného zásobníku (ve volné chvíli kdy
čeká na zprávu od dárce). Neaktivní/čekající procesor obsluhuje zprávy ve
smyčce tak, že jsou obslouženy všechny zprávy ve frontě a poté je procesor
uspán na 50ms a to opakuje do chvíle než přijme nějakou práci. Tím se
přepne se do stavu aktivní a počítá řešení.

Tento cyklus přijímání zpráv může být také ukončen přijmutím zprávy o
ukončení výpočtu. V takovém případě procesor ihned ukončí obsluhou zpráv,
ukončí algoritmus řešící úlohu a odesílá své nejlepší řešení Masterovi.
Master tyto řešení sekvenčně přijme od každého procesoru a porovnáním
zjistí konečné řešení.

% tabulka tak, ze velkost mriezky sa zobrazi cez dva spolocne riadky
	\begin{table}\centering
		\begin{tabularx}{\textwidth}{|X|X|X|X|X|X|X|X|}
			\hline                        
			\textbf{Veľkosť mriežky} & \textbf{Sieť} & \textbf{T(n,2)} & \textbf{T(n,4)} & \textbf{T(n,8)} & \textbf{T(n,16)} & \textbf{T(n,24)} & \textbf{T(n,32)} \\ \hline
			\multirow{2}{*}{\textbf{15x15}} & E 	& 27.67 & 11.97 & 16.15 & 15.33 & 25.60 & 27.23 \\ \cline{2-8}
			\textbf & I & 45.03 & 20.60 & 21.73 & 21.07 & 30.93 & 25.23  \\ \hline
			\multirow{2}{*}{\textbf{20x20}} & E 	& 126.07 & 68.83 & 42.43 & 47.37 & 70.87 & 52.17 \\ \cline{2-8}
			\textbf & I & 201.23 & 103.57 & 88.47 & 85.70 & 107.57 & 101.80  \\ \hline
			\multirow{2}{*}{\textbf{21x21}} & E 	& 420.13 & 200.17 & 68.63 & 58.00 & 67.97 & 92.83  \\ \cline{2-8}
			\textbf & I & 633.60 & 318.57 & 118.40 & 88.57 & 93.10 & 113.63  \\ \hline
			\multirow{2}{*}{\textbf{23x23}} & E 	& 78.67 & 13.30 & 25.73 & 40.63 & 44.17 & 50.47  \\ \cline{2-8}
			\textbf & I & 129.93 & 46.73 & 48.90 & 46.60 & 54.67 & 72.80  \\ \hline
		\end{tabularx}
	\caption{Trvanie paralelného výpočtu}
	\label{tab:par}
	\end{table}
	
	\begin{figure}[!h]\centering
	\includegraphics[scale=0.31,natwidth=578,natheight=343]{./vstup15x15.pdf}
	\caption{Paralelné zrýchlenie pre inštanciu problému 15x15}\label{fig:vstup15}
	\end{figure}

	\begin{figure}[!h]\centering
	\includegraphics[scale=0.31,natwidth=578,natheight=343]{./vstup20x20.pdf}
	\caption{Paralelné zrýchlenie pre inštanciu problému 20x20}\label{fig:vstup20}
	\end{figure}
	
	\begin{figure}[!h]\centering
	\includegraphics[scale=0.31,natwidth=578,natheight=343]{./vstup21x21.pdf}
	\caption{Paralelné zrýchlenie pre inštanciu problému 21x21}\label{fig:vstup21}
	\end{figure}	

	\begin{figure}[!h]\centering
	\includegraphics[scale=0.31,natwidth=578,natheight=343]{./vstup23x23.pdf}
	\caption{Paralelné zrýchlenie pre inštanciu problému 23x23}\label{fig:vstup23}
	\end{figure}

\section{Vyhodnotenie}

	Záver tejto správy by sme vyhradili analýze dosiahnutých výsledkov. Už letmým porovnaním tabuľiek \ref{tab:sek} a \ref{tab:par} je jasné, že v drvivej väčšine prípadov sme dosiahli stav, v ktorom paralelné spracovanie prebieha rýchlejšie ako sekvenčné riešenie. Zaujímavejšou otázkou preto zostáva hodnota zrýchlenia pre narastajúci počet procesorov pre všetky 4 analyzované riešenia.

	Prvý vstup o veľkosti mriežky 15x15, ktorého dĺžka sekvenčného spracovania nepresiahla 0.5 minúty, dosiahla najvyššiu hodnotu zrýchlenia už pre malý počet procesorov, konkrétne 4 pre siete Ethernet a InifiniBand súčasne - viz obrázok \ref{fig:vstup15}.
	
	Ďalšie vstupy o veľkosti mriežky 20x20 resp. 21x21, ktoré bežali na jednom procesore takmer 10 minút, dosiahli strop efektívnosti pri počte 8 resp. 16 procesorov. V porovnaní s menším vstupom 15x15 je tak zrejmé, že došlo k efektívnejšiemu využívaniu prostriedkov z dôvodu, že procesory pracovali nad dostatočne veľkým zásobníkom mriežok, vďaka čomu svoju prácu neukončili okamžite.
	
	Z hľadiska efektívnosti je nemenej zaujímavý prípad vstupu 23x23, ktorý na rozdiel od predchádzajúcich vstupov nebol vybraný náhodne. Naopak, pri jeho výbere sme hľadali mriežku s 1 veľkou hodnotou plochy dielčieho obdĺžnika. Naviac sme žiadali, aby hodnota plochy bola deliteľná minimálným počtom čísel, a teda \emph{shape} bol určený jednoznačne. Tieto predpoklady splnila mriežka obsahujúca hodnotu 121, ktorá sama zaberá približne 1/4 celkovej plochy. Vďaka nájdeniu takejto mriežky výsledný čas sekvenčného riešenia spĺňal hornú hranicu behu - 15 minút.
	
	Takto zvolený vstup 23x23 potom dosiahol pozoruhodné superlineárne zrýchlenie, napr. s použitím 2 procesorov došlo k zrýchleniu paralelného behu vyše 70-krát. Predpokladáme, že veľkú zásluhu na tejto hodnote má orezávanie neperspektívnych ciest na základe lokálne najlepšieho riešenia a posielanie najlepšieho známeho riešenia pri delení práce.
	
	Zaujímavosťou nášho riešenia je fakt, že s výnimkou jedného prípadu - veľkosť mriežky 15x15, beh na 32 procesoroch - sa ako rýchlejší ukázal štandard Ethernet. Vzhľadom na to, že nemáme dostatok informácií o konkrétnej implementácii siete InfiniBand, by sme z tohto výsledku nedokázali vyvodiť žiadne závery.
	
\section{Zdroje}

\begin{enumerate}

\item prednášky na eduxe k predmetu MI-PAR \begin{verbatim}https://edux.fit.cvut.cz/courses/MI-PAR.2 \end{verbatim}

\item správa k zadaniu Othello \begin{verbatim}http://www.michaltrs.net/cvut_fel/36par/36PAR-othello-zprava.pdf \end{verbatim}

\end{enumerate}

\end{document}
