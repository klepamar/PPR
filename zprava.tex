\documentclass[slovak]{article}
\usepackage[dvips]{graphicx}        % to include images
\usepackage{pslatex}	    % to use PostScript fonts
\usepackage[T1]{fontenc}
\usepackage[utf8]{inputenc}
\usepackage{pslatex}

\usepackage{tabularx} % tabulky na celu sirku strany
\usepackage{graphicx} %graphics files inclusion
\usepackage{pdfpages}

\begin{document}

\title{Semestrálna práca MI-PAR 2013/2014: \\[5mm] Paralelný algoritmus pre rozklad obdĺžnikov}
\author{Jakub Melezínek \\[2mm]Martin Klepáč}
\date{\today}

\maketitle

\section{Definícia problému}

Našou úlohou bolo vytvoriť program, ktorý implementuje usporiadanie obdĺžnikov v 2D mriežke so zachovaním minimálneho celkového obvodu týchto obdĺžnikov.

Obsah obdĺžnika je daný hodnotou uloženou v 2D mriežke, pričom tento element musí byť súčasťou obdĺžnika s daným obsahom.

Jednotlivé obdĺžniky sú vzájomne disjunktné až na spoločné vrcholy a hrany, pričom zjednotenie všetkých obdĺžnikov pokrýva pôvodnú mriežku.

Riešení, ako rozdeliť mriežku na jednotlivé obdĺžniky, môže byť viac -- v takom prípade hľadáme riešenie s minimálnym obvodom -- zároveň ale riešenie nemusí existovať.


\section{Formát vstupu, výstupu}

Formálne, vstup vyjadríme pomocou

\begin{itemize}

\item \emph{a, b} = prirodzené čísla predstavujúce rozmery mriežky

\item \emph{H[1..a][1..b]} = mriežka

\item \emph{n} = prirodzené číslo predstavujúce počet obdĺžnikov vo vnútri mriežky

\end{itemize}

Výstupom algoritmu je okrem celkového obvodu dielčích obdĺžnikov vyfarbená mriežka, t.j. mriežka, v ktorej každému elementu je priradené písmeno abecedy, ktoré jednoznačne identifikuje obdĺžnik, ktorého je bod súčasťou.


\section{Implementácia sekvenčného riešenia}

Primárny cieľ sekvenčného riešenia, na ktorom ďalej staviame v paralelnej implementácii, spočíva v nájdení a následnom prehľadaní celého stavového priestoru množiny potenciálnych riešení. 

Stavový priestor v našom prípade rozumieme množinu mriežok, v ktorých postupne spracúvame obdĺžniky s dvojicou parametrov

\begin{enumerate}

\item \emph{shape} - veľkosť obdĺžnika (dĺžka x šírka)

\item \emph{position} - pozícia v mriežke

\end{enumerate}

Očividne, \emph{shape} obdĺžnika je určený jeho plochou, zatiaľ čo \emph{position} je vlastnosťou mriežky a ostatných obdĺžnikov v nej existujúcich - napr. obdĺžnik nedokážem vložiť do mriežky, pokiaľ jeden z jeho rozmerov presahuje veľkosť mriežky alebo je okolie obdĺžnika posiate inými, už zafixovanými obdĺžnikmi.

Popis sekvečného riešenia v skratke: procesor zo zásobníka mriežok vezme obdĺžnik, zistí, či má \emph{shape} (v prípade negatívnej odpovede vygeneruje všetky dvojice \emph{a}, \emph{b} tak, aby ich súčin bol rovný očakávanej ploche). V ďalšom kroku procesor zistí, či daný obdĺžnik má stanovenú \emph{position} v mriežke - ak nie, na zásobník uloží všetky prípustné možnosti s ohľadom na veľkosť mriežky a susedné obdĺžniky. Procesor pokračuje do spracovania posledného obdĺžnika alebo do nájdenia prvého obdĺžnika, ktorý nemožno umiestniť do mriežky. V prípade, ak sa všetky obdĺžniky podarilo umiestniť do mriežky, procesor porovná výsledok s doterajším minimom a zo zásobníka vezme ďalšiu mriežku a opakuje takto popísanú akciu.

V nami implementovanom triviálnom sekvenčom riešení neuplatňujeme orezávanie neperspektívnych ciest - každá mriežka dobehne do konca v prípade existencie rozdelenia obdĺžnikov a až následne sa porovná výsledný obvod s doterajším minimom.

Trvanie sekvenčného výpočtu na výpočtovom klastri star.fit.cvut.cz je znázornené v tabuľke \ref{tab:sek}. Výsledný sekvenčný čas je určený ako aritmetický priemer trojice meraní.

	\begin{table}\centering
		\begin{tabularx}{\textwidth}{|X|X|X|}
			\hline                        
			\textbf{Veľkosť mriežky} & \textbf{T(n) [s]} \\ \hline
			\textbf{15x15} 	& 29 \\ \hline
			\textbf{20x20} 	& 473 \\ \hline
			\textbf{21x21} 	& 571 \\ \hline
			\textbf{23x23} 	& 983 \\ \hline
		\end{tabularx}
	\caption{Trvanie sekvenčného výpočtu}
	\label{tab:sek}
	\end{table}

\section{Príklad zadania a výsledku}

	\begin{figure}\centering
	\includegraphics[scale=0.40,natwidth=503,natheight=648]{./sequential.pdf}
	\caption{Príklad výstupu pre sekvenčnú úlohu}\label{fig:sek}
	\end{figure}

Program spúšťame s parametrom -f, ktorý udáva cestu k súboru obsahujúce vstupné dáta. Pre triviálny vstup o veľkosti mriežky 5x5 dostávame výstup zobrazený na obrázku \ref{fig:sek}.

\section{Implementácia paralelného riešenia}

	\begin{table}\centering
		\begin{tabularx}{\textwidth}{|X|X|X|X|X|X|X|}
			\hline                        
			\textbf{Veľkosť mriežky} & \textbf{T(n,2)} & \textbf{T(n,4)} & \textbf{T(n,8)} & \textbf{T(n,16)} & \textbf{T(n,24)} & \textbf{T(n,32)} \\ \hline
			\textbf{15x15} 	& & & & & &  \\ \hline
			\textbf 	& & & & & &  \\ \hline
			\textbf{20x20} 	& & & & & &  \\ \hline
			\textbf 	& & & & & &  \\ \hline
			\textbf{21x21} 	& & & & & &  \\ \hline
			\textbf 	& & & & & &  \\ \hline
			\textbf{23x23} 	& & & & & &  \\ \hline
			\textbf 	& & & & & &  \\ \hline
		\end{tabularx}
	\caption{Trvanie paralelného výpočtu}
	\label{tab:par}
	\end{table}

\end{document}
